\documentclass{article}

\usepackage{../preamble}
\standalonetrue

\pagestyle{fancy}
\fancyhf{}
\rhead{Section \thesection}
\lhead{PHYS 304 Lecture 13}
\rfoot{Page \thepage}


\title{PHYS 304 Lecture 13}
\author{Ashtan Mistal}
\date{November 02, 2021}

\begin{document}

\ifstandalone
\maketitle
\fi

\graphicspath{{./Lecture13/}}

Less than 35\% response rate, so very difficult to make firm conclusions Does seem to be consensus that using blackboard is preferred over mostly .pptx slides, so will continue with that, where it makes sense. Split roughly 50/50 on value of TPS exercises. Likely based on whether or not one does the pre-reading. TPS exercises designed to work well if pre-reading has been seriously reviewed. I will keep some, but likely less than at beginning of course. A fair number of people would like more iClicker questions. We will do that in tutorials and lectures, again, where it makes sense. Thanks to those who filled it out!


\section{Review of Key Points from Last Lecture}

A general quantum mechanical state, represented by the ket $\ket{S(t)}$, contains all the information it is possible to know about the dynamical evolution of a point particle moving in some time-independent potential. The wavefunction that we have so far associated with the state, $\Psi_s(x,t)$ - where I have added a subscript S to acknowledge the fact that the abstract state $\ket{S(t)}$ is in a sense “more fundamental” than the wavefunction used to represent it - is one of many possible wavefunctions that can be associated with the state $\ket{S(t)}$ .

The state $\ket{S(t)}$ can be equally well represented by any wavefunction, $\Psi_s(\zeta, t)$ , where $\zeta$ represents an eigen value of any physically-observable (and therefore Hermitian) operator associated with the classical dynamical variables in the problem ($x,p,E$, . . or any function $Q(x,p,E)$). i.e. solutions of the eigen value problem $\hat{\zeta} \ket{\zeta}$ . Explicitly,$\Psi_s(\zeta, t) = \braket{\zeta|S(t)}$, the inner product of the corresponding eigen state (not eigen function!!!), $\ket{\zeta}$ with the state  $\ket{S(t)}$


The components of the wavefunction (i.e. the values of the wavefunction at each eigen value) are multi-dimensional (usually infinite dimensional) generalizations of the duples used to itemize the “amount of” a 2D real vector along two independent unit vectors. Thus $\Psi_s(x,t)$ tells you how much of the state $\ket{S(t)}$ is at location $x$, $\Psi_s(x,t)$ tells you how much of the state $\ket{S(t)}$ has a momentum $p$, etc. Note the subtlety: It makes no sense to sat $\Psi_s(x,t)$ tells you how much of the wavefunction $\Psi_s(x,t)$ is at location $x$: a tautology. 

Thus there is nothing particularly special about working with the wavefunction $\Psi_s(x,t)$ associated with the choice $\zeta = x$ other than the fact that it has a particularly intuitive connection to classical notions of a localized particle, and the differential equation that can be used to explicitly solve for it, the Schrödinger equation, has a particularly simple form in the position basis. 

We showed that the eigenvalue equation for the position operator, $\hat{x} \ket{\zeta} = \zeta \ket{\zeta}$, (note that this is defined using states, not wavefunctions!) \textbf{that becomes $x' f(x') = \zeta f(x')$ int the position basis}, has eigenfunctions $f(x') = \delta(x - x')$ for a continuum of eigen states $\zeta = x$. 



\section{States, wavefunctions, and matrix representations}

\subsection{Example 1}

The eigenvalue equation for the position operator, $\hat{x} \ket{\zeta} = \zeta \ket{\zeta}$, (note that this is defined using states, not wavefunctions!) \textbf{\textit{that becomes $x' f(x') = \zeta f(x')$ in the position basis s...}}

\begin{itemize}
    \item \textit{how do we go from one to the other?}
    \item Starting from the generic state equation, project the bra $\bra{x'}$ onto both sides:
    $$\braket{x'|\hat{x}| \zeta} = \braket{x'|\zeta|\zeta}$$
    
    On the right hand side of this equation, the $\zeta$ between the bra and ket is just a number (an eigen value), and so $\braket{x'|\zeta|\zeta} = \zeta \braket{x'|\zeta}$, and $\braket{x'|\zeta}$ is just the wavefunction of the eigen state $\ket{\zeta}$ in the position basis (by definition). Call this function $\braket{x'|\zeta} = f(x')$
    
    Choose $\hat{\xi} = \hat x$
    
    $$\hat{x} \ket{x'} = x \ket{x}$$
    
    Project $\bra{x'}$ onto this equation:
    
    $\braket{x'|\hat{x}|x} - x \braket{x'|x}$, where $\braket{x'|x}$ is the wavefunction of eigenstate $\ket{x}$ in the position basis $\braket{x'|x} = f(x') = \psi_{\ket{x}}(x')$
    
    LHS:  $\braket{x'|\hat{x}|x} = \left( \bra{x'}|\hat{x} \right) \ket{x} = \braket{(\hat{x}^+ | x')|x}$
    $= \braket{(\hat{x} \ket{x'}) | x} = x; \braket{x'|x} = x' f(x')$
    
    LHS = RHS, and so $x' f(x') = x f'(x')$
    
    We showed last time the solution of this $f(x') = \delta(x -x')$
    
    This is is one specific example of how to go from a generic state equation (in bras and kets) to a differential equation in a specific dynamical variable.
    
    $$\int_{- \infty}^\infty x' \delta(x - x') g(x') dx' = \int_{= \infty}^\infty x \delta(x -x') g(x') dx' = x g(x)$$
    
    
\end{itemize}

\subsection{Example 2: States versus Wavefunctions}

"The components of the wavefunction (i.e. the values of the wavefunction at each eigen value) are multi-dimensional (usually infinite dimensional) generalizations of the duples used to itemize the “amount of” a 2D real vector along two independent unit vectors."

Another quantity of interest would be the expectation value of the position operator, which in Diract\footnote{bra ket notation} notation is:


To appreciate the wavefunction as an infinite dimensional vector (generalization of duple in 2d real vector space. consider the following:

$$\braket{x}_{S(t)} = \braket{S(t)|\hat{x}|S(t)}$$

Convert into form more familiar: Lets convert this into our familiar equation for the expectation value of the position operator using wavefunctions defined in the position basis.

Step 1: Insert the unity operator (twice) on the right hand side, to map the states into wavefunctions in the position basis. 
    
    $$\braket{\hat{x}}|_{S(t)} = \braket{S(t)| \underbrace{(\int \ket{x'} \bra{x'} dx')}_{\text{Unity operator}} |\hat{x}| \underbrace{(\int \ket{x''} \bra{x''} dx'')}_{\text{Unity operator}} |S(t)}$$
    
    Note that the integrals are from $-\infty$ to $\infty$. 
    
    The unity operator projects onto the position basis. We will get expressions involving position space. 
    
    $$\braket{x'}|_{S(t)} = \int_{- \infty}^\infty \int_{- \infty}^\infty dx' dx'' \braket{S(t)|x'} \braket{x'|\hat{x}|x''} \braket{x''|S(t)}$$
    
    But, we know that $\braket{x'' | S(t)} = \Psi_s (x'',t)$, and $\braket{S(t)|x'} = \Psi_s^* (x',t)$. WE also know that $\braket{x'|\hat{x}|x''} = \bra{x'}|(\hat{x} \ket{x''}) = x'' \braket{x'|x''} = x'' \delta(x' - x'')$. 
    
    Therefore, we get the following:
    
    $$\braket{x'}|_{S(t)} = \int_{- \infty}^\infty \int_{- \infty}^\infty dx' dx'' \Psi_s^* (x',t) x'' \delta(x' - x'') \Psi_s (x'',t) = \int_{- \infty}^\infty dx' \Psi_s^* (x',t) x' \Psi_s (x',t)$$
    
Now let's see how to convert $\braket{x'}|_{S(t)}$ into a matrix / vector equation. We already carried out the first step, by expressing it completely in the position basis. 


Replace $\int \int dx' dx''$ with $\sum_n \sum_m$ (shift to a discrete position space):

$$\braket{\hat{x}}_{\ket{S(t)}} = \sum_n \sum_m \braket{S(t)|x_m} \braket{x_m|\hat{x}|x_n} \braket{x_n|S(t)}$$

This should look familiar to:
$$\begin{bmatrix} . \\ . \\ . \\ \end{bmatrix} \begin{bmatrix} . & . & . \\ . & . & . \\ . & . & . \end{bmatrix} \begin{bmatrix} . & . & . \end{bmatrix}$$

Or, 

$$\begin{bmatrix} \psi^* (x_m) \\ \vdots \\ \vdots \\ \vdots \end{bmatrix} \begin{bmatrix} x_1 & \dots & \dots & \dots \\ \vdots & x_2 & \ddots & \vdots \\ \vdots & \ddots & \ddots & \vdots \\ \vdots & \dots & \dots & x_n \end{bmatrix} \begin{bmatrix}  \psi_s (x_n) & \dots & \dots & \dots \end{bmatrix}$$

\section{Momentum Basis}

A logical first step? Operate on $\ket{S(t)}$ with the unity operator expanded in the momentum basis

$$\ket{S(t)} = \int \ket{p} \braket{p|dp|S(t)} = \int dp \braket{p|S(t)} \ket{p}$$

$$\ket{S(t)} = \int dp \left( \int dx' \braket{p | x'} \braket{x' S(t)} \right) \ket{p}$$

$$ = \int dp  \left( dx' \braket{p|x'} \Psi_s(x',t) \right) \ket{p}$$

What is $\braket{p|x'}$? $\braket{p|x'} = \braket{x'|p}^*$, but we know that $\braket{x'|p}$ is the wavefunction in the position basis of the eigen state of the momentum operator with eigenvalue $p$. How do we figure that out? 

We know how to solve that problem in the position basis using space operators and wavefunctions:

$$-i \hbar \frac{d}{dx'} \braket{x'|p} = p \braket{x'|p}$$

We know the solution, to a normalization constant, is

$$\braket{c'|p} = e^{- \frac{p}{\hbar} x'}$$

Leave it to you to show that the normalization constant is $\frac{1}{\sqrt{2 \pi \hbar}}$, and therefore 

$$\braket{c'|p} = \frac{1}{\sqrt{2 \pi \hbar}} e^{- \frac{p}{\hbar} x'}$$

Substitute into $\ket{S(t)} = \int dp ( \int dx' \braket{p|x'} \Psi_s(x',t)) \ket{p}$, and we get

$$\ket{S(t)} = \int dp \left( \int dx' \frac{1}{\sqrt{2 \pi \hbar}} e^{- \frac{p}{\hbar} x'} \right) \ket{p}$$

So, we can solve for the expansion coefficients knowing $\Psi_s(x',t)$. 

This looks awfully like $\Psi(x,t) = \sum_{n=1}^\infty c_n \psi_n(x)$ with $c_n = \int_0^a \psi_n^*(x) \Psi(x,t) dx$. How do we prove that they are equivalent?


$$\ket{S(t)} = \int dp \left( \int dx' \frac{1}{\sqrt{2 \pi \hbar}} e^{- \frac{p}{\hbar} x'} \right) \ket{p}$$

SO, project $\bra{x}$ onto both sides, to yield

$$\braket{x|S(t)} = \int dp \left( \int dx' \frac{1}{\sqrt{2 \pi \hbar}} e^{- \frac{p}{\hbar} x'} \right) \braket{x|p}$$

Or, equivalently, 

$$\Psi_x(x,t) = \int dp \left( \int dx' \frac{1}{\sqrt{2 \pi \hbar}} e^{- \frac{p}{\hbar} x'} \right) \frac{1}{\sqrt{2 \pi \hbar}} e^{- \frac{p}{\hbar} x}$$

\subsection{Take aways?}


When actually having to explicitly solve an equation in Dirac notation, you often have to resort to translating it into a differential equation involving wavefunctions, then solve the differential equation. This “translation” process is facilitated by knowing where and how to insert the unity operator expressed in a well-chosen basis. Often, but not always, the position basis is chosen to solve the differential equations. (our starting point in this course!)



\end{document}