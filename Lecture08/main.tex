\documentclass{article}

\usepackage{../preamble}
\standalonetrue

\pagestyle{fancy}
\fancyhf{}
\rhead{Section \thesection}
\lhead{PHYS 304 Lecture 08}
\rfoot{Page \thepage}


\title{PHYS 304 Lecture 08}
\author{Ashtan Mistal}
\date{!!!}

\begin{document}

\ifstandalone
\maketitle
\fi

\graphicspath{{./Lecture08/}}

\section{Review of key points from last day}

Solutions for the harmonic oscillator potential are slightly more challenging, mathematically, than the infinite square well, but still fairly straight forward. The results are qualitatively similar in that there are discrete energies at which stationary state solutions exist, and the corresponding wavefunctions have an increasing number of nodes and anti-nodes to them as their energy increases. The spacing between energy levels is however fixed. Some distinctions between the quantum states and the classical solution are evident: there is some probability for the quantum particle to be found outside of the classical range of the particle, and for low energy states, the probability distribution $|\psi_n(x)|^2$ is dramatically different than the classical state probability distribution.  All stationary states have a probability distribution that vanishes at classically-allowed positions, but for large n, any real position measurement would average over rapid oscillations yielding close to the classical result.

\section{Free Particle}

Follow the "recipe" - \textbf{that you should not be intimately familiar with} - for constructing and analyzing the dynamical properties of non-stationary states of particles. 

Note that the basic set of steps that we are going to follow is identical to what we did for previous examples. GO back to Lectures 4 and 5 ("The most important lectures in the course"). Recall: why is it that we need to construct and analyze non-stationary states in order to study dynamics?

In order to master the first half of the course, you must thoroughly understand everything mentioned above. 

\subsection{Activity 1}

What is the general form of an apparent "stationary state" solution of the Schrodinger equation for a particle in free space, and why can't it be normalized?

Why is this distinct from the reason that we "threw out" some of the mathematically-allowed solutions to the time-independent Schrodinger equation when we dealt with the harmonic oscillator problem, or why don't we consider stationary solutions for this free particle case when $E_k < 0$?


Because the wavefunction itself doesn't diverge, it is only the fact that it doesn't go to zero at +/- infinity that is the problem…hence can be “fixed” 




\end{document}